
Die vorliegende Hausarbeit beschäftigt sich mit dem Controller Area Network (CAN) und dessen Telegrammen. Ziel der Arbeit ist es, die Struktur und Funktionsweise von CAN-Telegrammen zu erläutern und die verschiedenen Aspekte des CAN-Bussystems zu untersuchen. Die Arbeit beginnt mit einer detaillierten Beschreibung der beiden grundlegenden Telegrammtypen: den Standard-CAN-Frames nach der Spezifikation 2.0A und den Extended-CAN-Frames nach der Spezifikation 2.0B. Hierbei wird insbesondere auf den Unterschied in der Länge des CAN-Identifiers und der verfügbaren logischen Adressen eingegangen.

Anschließend wird der Buszugriffsmechanismus im CAN-Bus behandelt, wobei der Ablauf der bitweisen Arbitrierung und die Bedeutung der Priorität der Nachrichten thematisiert werden. Ein weiterer wichtiger Bestandteil der Arbeit ist das Fehlermanagement im CAN-System. Hierzu werden die verschiedenen Fehlerarten und die Funktionsweise der Fehlererkennung sowie die Maßnahmen zur Fehlerbehandlung erläutert.

Abschließend wird das CAN-Protokoll im Kontext des ISO/OSI-Modells betrachtet, wobei der Fokus auf der physikalischen und der Sicherungsschicht liegt. Diese Struktur ermöglicht es, die Funktionsweise des CAN-Bussystems in einem größeren Netzwerkzusammenhang zu verstehen. Die Arbeit ist in mehrere Abschnitte unterteilt, die aufeinander aufbauen und die relevanten Themen des CAN-Busses detailliert darstellen.
