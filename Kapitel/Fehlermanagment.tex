
\subsection{Arten von Fehlern}
Ein Bussystem sendet eine große Menge an Daten, dabei kann es zu Fehlern in der Übertragung kommen.
Der CAN benutzt hierfür das "Data Link Layer" welches in standardisierten CAN-Chips implementiert ist.
Es gibt verschiedene Fehler diese werden folgend aufgeführt:
Bit Fehler: Bei der Überprüfung der Nachricht gibt es einen Falschen Wert.

Bit-Stuffing-Fehler: In CAN wird eine Bitfolge von mehr als 5 gleichen Bits als Fehler erkannt.

Es sei den es handelt sich um das EOF (End of Frame)

Fehler: Die CRC-Prüfsumme stimmt nicht mit der berechneten übereinstimmen.

Format Fehler: Es liegt das falsche Datenformat vor.

Acknowledgement Fehler: Ein Sender empfängt von keinem Busteilnehmer ein dominates Bit im Acknowledgement
-Slot
\subsection{Funktionsweise des Fehlermanagment}
Der CAN behandelt Fehler immer nach dem gleichen Prinzip dieses wird im folgenden aufgeführt.
Zunächst muss einer der oben genannten Fehler genannt werden dafür sind diese Fehler bei jedem Busteilnehmer
hinterlegt. Nachdem ein Busteilnehmer einen Fehler erkennt hat, sendet er einen Error Frame (mehr als
5 gleiche Bits) das die Stuffing-Regel verletzt. Dies dient zur Übermittlung eines Fehlers an alle anderen
Busteilnehmer, dadurch verwerfen alle Busteilnehmer das zuvor empfangene Telegramm. Anschließend
werden bei den Busteilnehmern die Zähler TEC(Transmission Error Counter) oder REC(Receive Error Counter)
beeinflusst. Abschließend wird das Telegramm erneut gesendet.
Die oben genannten Zähler dienen dazu um dauerhafte Fehler durch defekte Hard- oder Software zu 
verhindern.
Jeder Busteilnehmer/Knoten kann drei Zustände annehmen je nachdem welchen Wert die Zähler haben.
Diese Zustände bestimmen welche Priorität eines Busteilnehmers hat und zu bestimmen ob ein Busteilnehmer
Fehlermeldungen senden darf oder nicht.
Der erste Zustand ist "Active Error"(TEC \& REC < 128), dieser berechtigt einen CAN-Knoten ein dominantes
Error Frame zu senden sollte er einen Fehler feststellen.
Der zweite Zustand ist "Passive Error"(TEC || REC >127), dieser sorgt dafür das der CAN-Knoten nur noch
rezessive Error Frames zu senden um zu verhindern das der CAN-Knoten den Bus weiter mit Fehlermeldungen
behindert.
Der letzte Zustand ist "Bus Off"(TEC > 255), dieser trennt den Busteilnehmer von der Kommunikation im
Bussystem. Dieser Zustand kann nur verlassen werden durch einen Soft- oder Hardware Reset.

Wie oben erwähnt werden die Zähler oben beeinflusst, dies geschieht mittels verschiedener Algorithm
um den Fehler beim richtigen Knoten zuzuordnen.
