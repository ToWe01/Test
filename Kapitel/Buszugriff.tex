Das CAN-Bussystem ist ein Multi-Master-System mit mehreren Busteilnehmern. 
Daher muss der Zugriff auf den Bus geregelt werden, um zu verhindern, 
dass der Bus dauerhaft besetzt ist. 
Hierfür regelt das CAN-Bussystem den Zugriff mithilfe der zerstörungsfreien, bitweisen Arbitrierung.

Wie oben erläutert, befindet sich in jedem CAN-Telegramm ein Arbitrierungsfeld, das aus dem Startbit, 
einem Identifier und einem RTR-Bit (Remote Transmission Request) besteht. 
Für den Buszugriff sind jedoch nur das Startbit und der Identifier relevant, da diese die Priorität 
der Nachricht festlegen. Dabei gilt: Eine Null ist dominant, und eine Eins ist rezessiv. 
Das RTR-Bit wird nur dann genutzt, wenn ein Teilnehmer Daten von einem anderen Teilnehmer anfordert.

Die Priorität der Nachricht wird durch die Wertigkeit des Identifiers bestimmt. 
Nachrichten mit kleineren Identifiern haben höhere Priorität und setzen sich bei gleichzeitigem 
Zugriff auf den Bus durch. Da der CAN-Bus durch eine Null in einem Bit als „dominant“ belegt wird, 
führt ein niedrigerer Identifier (mit mehr Nullen) automatisch zu einer höheren Priorität.

\subsection{Ablauf der Arbitrierung}
Wenn zwei oder mehr Teilnehmer gleichzeitig auf den Bus zugreifen, 
folgt der Zugriff einem klaren Ablauf: Beide Teilnehmer senden zunächst ihr 
Startbit (z. B. 0 = dominant), wodurch der Bus auf den dominanten Zustand eingestellt wird. 
Beide Teilnehmer lesen den Zustand aus und senden weiter, solange die Bits, die sie senden, 
mit dem empfangenen Wert übereinstimmen. Auf diese Weise erkennen sie nicht, 
dass ein anderer Teilnehmer ebenfalls sendet.

Dieser Prozess der bitweisen Arbitrierung setzt sich fort, bis es zu einer Abweichung in den 
gesendeten Bits kommt. In diesem Moment stellt der Teilnehmer, der eine Eins (rezessiv) gesendet hat, 
fest, dass sein Signal vom anderen Teilnehmer (der eine Null gesendet hat) „überstimmt“ wurde. 
Der Teilnehmer mit dem rezessiven Bit bricht daraufhin die Übertragung ab und schaltet in den 
Empfangsmodus, während der Teilnehmer mit dem dominanten Bit seine Nachricht ohne Unterbrechung 
weitersendet.

Dieses Verfahren wird als „zerstörungsfrei“ bezeichnet, 
da der Teilnehmer mit der höheren Priorität (der das dominante Signal gesendet hat) seine 
Übertragung nicht neu starten muss. Der abgebrochene Teilnehmer bleibt im Empfangsmodus und 
kann die Nachricht weiterhin lesen, da sie auch für ihn relevant sein könnte. 
Er muss das Telegramm nicht neu starten, da es den gleichen Anfang hat wie seine eigene Nachricht.

\subsection{Sicherheit und Fehlertoleranz}
Diese bitweise Arbitrierung macht das CAN-Bussystem besonders sicher und fehlertolerant. 
Da die Arbitrierung zerstörungsfrei erfolgt, gehen keine Daten verloren, 
und konkurrierende Teilnehmer können ihre Nachrichten sofort erneut senden, 
sobald der Bus wieder frei ist. Dadurch wird eine hohe Datenintegrität und Effizienz gewährleistet.

\subsection{Physikalische Einstellung des Buszustands}
Voraussetzung für dieses System ist die richtige Einstellung des physikalischen Treibers, 
sodass 0 als dominant und 1 als rezessiv erkannt wird. 
In den meisten standardisierten CAN-CPUs ist die Null als dominant eingestellt, 
sodass Nachrichten mit kleineren Identifiern (mehr Null-Bits) automatisch eine höhere 
Priorität haben.
