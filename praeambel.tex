%%%Präambel zum laden benötigter Packages%%%

%Anpassung des Dokuments auf die deutsche Sprache
\usepackage[ngerman]{babel}

%Links und Verweise innerhalb des Dokuments erzeugen

\usepackage[breaklinks, pdftex]{hyperref}

%Geometrische Form des Dokuments
\usepackage[a4paper, left=3cm, right=2cm, top=3.5cm , bottom=2.5cm]{geometry}

%Paket zum einbinden von Bildern und Grafiken
\usepackage{graphicx}

%Für zwei Bilder nebeneinander
\usepackage{subfigure}

%Matheumgebung
\usepackage{amsmath}

%Mathematische Symbole
\usepackage{amssymb}

%Transformationszeichen
\usepackage{trfsigns}

%Fancy-Mathe-Buchstaben (z.B. Laplace L)
\usepackage{mathrsfs}

%Paket zum Darstellen von SI-Einheiten
\usepackage{siunitx}

%Sourcecode Printer in Latex
\usepackage{listings}

%Paket zum einbinden von Tikz-Dateien(Matlab)
\usepackage{pgfplots}
\pgfplotsset{compat=1.16}


%Ausgabeformat der PDF
\usepackage[T1]{fontenc}

%Eingabeformat der Latex Datei
\usepackage[utf8]{inputenc}

%Horizontale Linien innerhalb einer Tabelle
\usepackage{booktabs}

%Tabelle mit festen Breiten 
\usepackage{tabularx}

\usepackage{csquotes}
 
%Zeilenabstand einstellen 
\usepackage[onehalfspacing]{setspace}
%Befehle zum umschalten des Zeilenabstandes 
%\onehalfspacing             % anderthablfacher Zeilenabstand 

%Absatzabstand einstellen
\setlength{\parskip}{6pt}
%Absatzeinzug einstellen
\setlength{\parindent}{0pt}

%Erstellen von Abkürzungen+ Verzeichnis
\usepackage[withpage]{acronym}
%Optionen:	footnote: Fußnoten der kurzform
%			nohyperlinks: keine Verlinkungen
%			printonlyused: nur verwendete Abkürzungen
%			smaller: verkeinerung der Anzeige der Abkürzung
%			dua: Langform der Abkürzung angezeigt
%			nolist: Keine Übersciht der Abkürzungen
%			withpage: Gibt die Seitenzahl mit an



%Für den Float Specifier  
\usepackage{float}

		
%Bibliotheksverwaltung und Zitieren
\usepackage[backend=biber,url=false, sorting=none]{biblatex}

%Definition eines Literatur und eines Quellenverzeichnisses
\defbibheading{AL}{\section*{Literatur}}
\defbibheading{QL}{\section*{Quellen}}
\bibliography{Meine_Bibliothek.bib}
\defbibfilter{AL}{\not\keyword{Quellen}}
\defbibfilter{QL}{\keyword{Quellen}} %Quellen benötigen das Keyword "Quellen" in Jabref!
%Änderung in Buchstaben mit dem Feld "Shorthand"!
%Entfernen der Anführungszeichen im Titel und Hinzufügen eines Doppepunktes

\DeclareFieldFormat[article]{title}{{#1}}
\DeclareFieldFormat[misc]{title}{#1}
\DeclareFieldFormat[article]{title}{{#1}}
\DeclareFieldFormat[MastersThesis]{title}{#1}

\DefineBibliographyStrings{german}{% 
        andothers = {{et\,al\adddot}},             
        editor = {(Hrsg.)}, 
        editors = {(Hrsg.)}}
        
   

%Einstellung neue Längen
\newlength\Bildbreite
\setlength\Bildbreite{0.9\textwidth}

\numberwithin{equation}{section} 
\renewcommand{\theequation}{\arabic{section}.\arabic{equation}}


\usepackage[labelfont={bf,sf},font={small},%
  labelsep=space]{caption}